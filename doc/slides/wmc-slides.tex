\documentclass[t,pdf]{beamer}
\mode<presentation>{}


\usecolortheme[RGB={196, 30, 58}]{structure}


\usepackage{color}
\usepackage{animate}
\usepackage{tikz}
\usetikzlibrary{shadings,shadows}
\usetikzlibrary{shapes, arrows}
\usetikzlibrary{decorations.pathreplacing,angles,quotes}
\usetikzlibrary{calc}
\usetikzlibrary{positioning}
\usepackage{pgfplots}
\usepackage{graphicx}
\usepackage{adjustbox}
\usepackage{scrextend}
\usepackage{booktabs}


\usepackage{hyperref}
%% Colored hyperlink 
\newcommand{\cref}[2]{\href{#1}{\color{blue}#2}}
%% Colored hyperlink showing link in TT font
% \newcommand{\chref}[1]{\href{#1}{\small\tt \color{blue}#1}}
\newcommand{\hcref}[1]{\cref{#1}{\small\tt #1}}

\newcommand{\obar}[1]{\overline{#1}}
\newcommand{\nil}{\bot}
\newcommand{\bitem}{\item[$\bullet$]}
\newcommand{\opos}[1]{#1}
\newcommand{\oneg}[1]{\overline{#1}}
\newcommand{\xnot}{\oneg{x}}
\newcommand{\lit}{\ell}
\newcommand{\approximate}[1]{\hat{#1}}
\newcommand{\approxv}{\approximate{v}}
\newcommand{\approxw}{\approximate{w}}
\newcommand{\assign}{\alpha}
\newcommand{\modelset}{{\cal M}}
\newcommand{\aerror}{\delta}
\newcommand{\digitprecision}{\Delta}


\newcommand{\inputformula}{\phi}
\newcommand{\compiledformula}{\psi}

\usepackage{booktabs,colortbl}

\definecolor{redorange}{rgb}{0.878431, 0.235294, 0.192157}
\definecolor{lightblue}{rgb}{0.552941, 0.72549, 0.792157}
\definecolor{clearyellow}{rgb}{0.964706, 0.745098, 0}
\definecolor{clearorange}{rgb}{0.917647, 0.462745, 0}
\definecolor{mildgray}{rgb}{0.54902, 0.509804, 0.47451}
\definecolor{softblue}{rgb}{0.643137, 0.858824, 0.909804}
\definecolor{bluegray}{rgb}{0.141176, 0.313725, 0.603922}
\definecolor{lightgreen}{rgb}{0.709804, 0.741176, 0}
\definecolor{redpurple}{rgb}{0.835294, 0, 0.196078}
\definecolor{midblue}{rgb}{0, 0.592157, 0.662745}
\definecolor{clearpurple}{rgb}{0.67451, 0.0784314, 0.352941}
\definecolor{browngreen}{rgb}{0.333333, 0.313725, 0.145098}
\definecolor{darkestpurple}{rgb}{0.396078, 0.113725, 0.196078}
\definecolor{greypurple}{rgb}{0.294118, 0.219608, 0.298039}
\definecolor{darktruqoise}{rgb}{0, 0.239216, 0.298039}
\definecolor{darkbrown}{rgb}{0.305882, 0.211765, 0.160784}
\definecolor{midgreen}{rgb}{0.560784, 0.6, 0.243137}
\definecolor{darkred}{rgb}{0.576471, 0.152941, 0.172549}
\definecolor{darkpurple}{rgb}{0.313725, 0.027451, 0.470588}
\definecolor{darkestblue}{rgb}{0, 0.156863, 0.333333}
\definecolor{lightpurple}{rgb}{0.776471, 0.690196, 0.737255}
\definecolor{softgreen}{rgb}{0.733333, 0.772549, 0.572549}
\definecolor{offwhite}{rgb}{0.839216, 0.823529, 0.768627}

\definecolor{mediumgreen}{RGB}{20,140,20}
\definecolor{mediumblue}{RGB}{20,20,140}
\definecolor{medgreen}{rgb}{0.34, 0.65, 0.34}

\newcommand{\red}[1]{\textcolor{red}{#1}}
\definecolor{darkred}{RGB}{180,0,0}
\newcommand{\darkred}[1]{\textcolor{darkred}{#1}}
\definecolor{dominocolor}{RGB}{0,0,128}
\definecolor{darkgreen}{RGB}{0,180,0}
\definecolor{darkgray}{RGB}{128,128,128}
%%\newcommand{\ground}{blue}

%%\newcommand{\ft}[1]{\frametitle{#1}}
%%\newcommand{\ig}[2]{\includegraphics[#1]{#2}}

\definecolor{xred}{rgb}{0.77, 0.12, 0.23}
\definecolor{xgreen}{rgb}{0.3, 0.6, 0}
\definecolor{xblue}{rgb}{0., 0.25, 1}

\newcommand{\btext}[1]{\textcolor{xblue}{#1}}
\newcommand{\rtext}[1]{\textcolor{xred}{#1}}
\newcommand{\gtext}[1]{\textcolor{xgreen}{#1}}
\newcommand{\wtext}[1]{\textcolor{white}{#1}}

\title{\huge Numerical Considerations \\ for Weighted Model Counting}
%\subtitle{}
\author{Randal E. Bryant}
\institute{\includegraphics[height=50pt]{CMU_Logo}}

\date{\textcolor{black}{Workshop on Counting, Sampling, and Synthesis 2025}}


\setbeamertemplate{footline}
{
	\leavevmode%
	\hbox{%
	\begin{beamercolorbox}[wd=0.35\paperwidth,ht=2.25ex,dp=1ex,center]{author in head/foot}%
%%	\tiny {\url{http://www.cs.cmu.edu/~bryant}}
%%			\vspace{4pt}
	\end{beamercolorbox}%
	\begin{beamercolorbox}[wd=0.45\paperwidth,ht=2.25ex,dp=1ex,center]{author in head/foot}%
	\end{beamercolorbox}%
	\begin{beamercolorbox}[wd=0.2\paperwidth,ht=2.5ex,dp=1ex,right]{date in head/foot}%
		\structure{\scriptsize \insertframenumber{} / \inserttotalframenumber\hspace*{3ex}}
		\vspace{3pt}
	\end{beamercolorbox}}%
	\vskip0pt%
}

\beamertemplatenavigationsymbolsempty

\begin{document}

\begin{frame}
	\titlepage

\small\url{http://www.cs.cmu.edu/~bryant}
\end{frame}

\begin{frame}

  \frametitle{Weighted Model Counting Definition}

  \begin{center}
\begin{tikzpicture}[scale=0.035]
  \node [left] at (0,25) {$\inputformula$};
  \node [right] at (220,25) {$w(\inputformula)$};
  \node [left] at (0,-20){$w$};

  \draw[fill=structure] (60,0) rectangle (160,50);

%  \draw[line width=2pt] (80,25) [-latex] -- (140,25);
  \draw[line width=2pt] (0,25) [-latex] -- (60,25);
  \draw[line width=2pt] (160,25) [-latex] -- (220,25);
  \draw[line width=2pt] (0,-20) -- (110,-20) [-latex] -- (110,0);

% \node [left] at (10,55) {\begin{tabular}{c} Input\\Formula \end{tabular}};

  \node [left] at (10,52) {Input};
  \node [left] at (10,40) {Formula};

  \node [left] at (10, 7) {Weight};
  \node [left] at (10,-5) {Assignment};

  \node [right] at (210,52) {Weighted};
  \node [right] at (210,40) {Count};


  \node[white] at (110,37) {Weighted};
  \node[white] at (110,25) {Model};
  \node[white] at (110,13) {Counter};
\end{tikzpicture}

\end{center}

\vskip -10pt
  {\bf Given}

  \begin{itemize}
  \item CNF formula $\inputformula$ over variables $X$
    \begin{itemize}
      \bitem Set of satisyfing assignments (\emph{models}) $\modelset(\inputformula)$
      \bitem Each model $\assign$ is set of literals 
    \end{itemize}
  \item Weight assignment $w(x), w(\obar{x}) \in \mathbb{Q}$ for each $x \in X$
  \end{itemize}

  {\bf Compute}
  \begin{displaymath}
  w(\inputformula) \;\; \doteq \;\; \sum_{\assign \in \modelset(\inputformula)} \;\;\prod_{\lit \in \assign} w(\lit) 
  \end{displaymath}

\end{frame}

\begin{frame}
  \frametitle{Weighted Model Counting Computation}

\begin{center}
  \Large \emph{What numerical representation should be used?}
\end{center}

\bigskip
\begin{minipage}[t]{0.48\textwidth}
{\bf Rational Arithmetic}
\begin{itemize}
  \bitem Represent $v \in \mathbb{Q}$ as $v = a/b$
  \bitem Represent $a$ and $b$ as multiprecision integers
  \bitem Compute exact result
  \bitem High time and space cost
\end{itemize}
\end{minipage}
\begin{minipage}[t]{0.48\textwidth}
{\bf Floating-Point Arithmetic}
\begin{itemize}
  \bitem Approximate $v \in \mathbb{Q}$ as $\approxv = (-1)^s \times f \times 2^{e}$
  \bitem Represent $f$ and $e$ with fixed number of bits
  \bitem Low time and space cost
\end{itemize}
\end{minipage}

\bigskip
\only<2>{
\textbf{Claim:}

\begin{itemize}
\item  Floating-point preferred
\item But only if desired precision can be guaranteed
\end{itemize}
}

\end{frame}


\begin{frame}
  \frametitle{Approximation Metrics}

For value $v$ and its approximation $\approxv$:

  \begin{displaymath}
\aerror[\approxv, v] \;\; \doteq \;\; \left\{ \begin{array}{lll}
  \frac{|\approxv - v|}{|v|}  & v \not = 0 & \textsf{Relative} \; \textsf{error}\\[0.5em]
  0 & v  = \approxv = 0 & 0 \approx 0\\[0.5em]
  1 & v = 0 \; \textsf{and} \; \approxv \not = 0 & 0 \not \approx \epsilon,\;\textsf{for} \;\epsilon \not = 0
  \end{array} \right.
\end{displaymath}

\medskip

\only<2>{
\textbf{Digit Precision}

\begin{displaymath}
  \digitprecision(\approxv, v) \;\; \doteq \;\; \max(0, -\log_{10} \aerror[\approxv, v])
\end{displaymath}

\begin{itemize}
 \item Number of trustworthy digits in decimal representation
\item  $= \infty$ when $v = \approxv$
\item $= 0$ when $v \not = 0$ and $\approxv = 0$
\item $= 0$ when $v = 0$ and $\approxv \not = 0$
\end{itemize}
} % only
\end{frame}

\begin{frame}

  \frametitle{Weighted Model Counting: Floating-Point Approximation}


\begin{tikzpicture}[scale=0.035]
  \node [left] at (0,75) {$D$};

  \node [left] at (0,25) {$\inputformula$};
  \node [right] at (220,25) {$\approxw(\inputformula)$};
  \node [left] at (0,-20){$w$};

  \draw[fill=structure] (60,0) rectangle (160,50);

%  \draw[line width=2pt] (80,25) [-latex] -- (140,25);
  \draw[line width=2pt] (0,75) -- (110,75) [-latex] -- (110,50);

  \draw[line width=2pt] (0,25) [-latex] -- (60,25);
  \draw[line width=2pt] (160,25) [-latex] -- (220,25);
  \draw[line width=2pt] (0,-20) -- (110,-20) [-latex] -- (110,0);

% \node [left] at (10,55) {\begin{tabular}{c} Input\\Formula \end{tabular}};

  \node [left] at (10,102) {Target};
  \node [left] at (10,90) {Precision};


  \node [left] at (10,52) {Input};
  \node [left] at (10,40) {Formula};

  \node [left] at (10, 7) {Weight};
  \node [left] at (10,-5) {Assignment};

  \node [right] at (210,64) {Approximate};
  \node [right] at (210,52) {Weighted};
  \node [right] at (210,40) {Count};


  \node[white] at (110,37) {Weighted};
  \node[white] at (110,25) {Model};
  \node[white] at (110,13) {Counter};
\end{tikzpicture}

\bigskip

  {\bf Approximation}

  \begin{itemize}
  \item Set target precision $D$
  \item Compute approximation $\approxw(\inputformula)$
  \item Require $\digitprecision(\approxw, w) \geq D$
  \end{itemize}

\end{frame}

\begin{frame}
  \frametitle{Choosing Target Precision}

\bigskip
  $D = 1$
  \begin{itemize}
    \item $10\%$ accuracy
    \item Good enough for 2020 Weighted Model Competition

  \end{itemize}

\bigskip
  $D = 3$
  \begin{itemize}
    \item $0.1\%$ accuracy
    \item Good enough for 2024 Weighted Model Competition
  \end{itemize}

\bigskip
\only<2>{
  $D = 30$
  \begin{itemize}
  \item Distance from Earth to Alpha Centauri $\approx 4.37$ light years.
  \item Hydrogen atom diameter $\approx 106$ picometers.
  \item $D=30$ can represent distance from Earth to Alpha Centauri to nearest picometer
  \end{itemize}
}

\end{frame}

\begin{frame}

\frametitle{Using Knowledge Compilation}

  \begin{center}
\begin{tikzpicture}[scale=0.035]
  \node [left] at (0,75) {$D$};
  \node [left] at (0,25) {$\inputformula$};
  \node [right] at (220,25) {$\approxw(\inputformula)$};
  \node [left] at (0,-20){$w$};

  \node at (110,25) {$\psi$};

  \draw[fill=structure] (20,0) rectangle (80,50);
  \draw[fill=structure] (140,0) rectangle (200,50);

  \draw[line width=2pt] (0,25) [-latex] -- (20,25);
%  \draw[line width=2pt] (80,25) [-latex] -- (140,25);

  \draw[line width=2pt] (80,25) [-latex] -- (100,25);  
  \draw[line width=2pt] (120,25) [-latex] -- (140,25);

  \draw[line width=2pt] (200,25) [-latex] -- (220,25);

  \draw[line width=2pt] (0,75) -- (170,75) [-latex] -- (170,50);
  \draw[line width=2pt] (0,-20) -- (170,-20) [-latex] -- (170,0);

% \node [left] at (10,55) {\begin{tabular}{c} Input\\Formula \end{tabular}};
  \node [left] at (10,102) {Target};
  \node [left] at (10,90) {Precision};


  \node [left] at (10,52) {Input};
  \node [left] at (10,40) {Formula};

  \node [left] at (10, 7) {Weight};
  \node [left] at (10,-5) {Assignment};

  
  \node at (110,60) {decision-DNNF};
  \node at (110,48) {Formula};

  \node [right] at (210,64) {Approximate};
  \node [right] at (210,52) {Weighted};
  \node [right] at (210,40) {Count};


  \node[white] at (50,31) {Knowledge};
  \node[white] at (50,19) {Compiler};

  \node[white] at (170,37) {Weighted};
  \node[white] at (170,25) {Model};
  \node[white] at (170,13) {Counter};
\end{tikzpicture}
\end{center}

  \begin{itemize}
  \item Convert input formula $\inputformula$ into logically equivalent formula $\psi$
  \item $\psi$ has form that makes weighted counting tractable
  \end{itemize}

\end{frame}


\begin{frame}
  \frametitle{Experimental Exploration}

\medskip  

  \textbf{Benchmark Formulas}
  \begin{itemize}
  \item 200 formulas from 2024 Weighted Model Competition
  \item Compile with D4 version 2 knowledge compiler
    \begin{itemize}
    \bitem Succeessfully compiled 100 within 3600 seconds on machine with 64~GB memory
    \bitem 33 -- 325,113 variables
    \bitem 209 -- $1.66\times 10^9$ binary operations in compiled representation
    \end{itemize}
  \end{itemize}

\medskip

    \textbf{Counting}
    \begin{itemize}
    \item Use weights included in CNF file
    \item Exact result using GMP~MPQ rational arithmetic library
      \begin{itemize}
       \bitem Memory limit exceeded for two formulas
      \end{itemize}
    \item Floating-point using GMP~MPF software floating-point library
      \begin{itemize}
       \bitem Using fraction size $p = 128$
      \end{itemize}
    \end{itemize}

\medskip

  \textbf{Evaluation}
  \begin{itemize}
  \item Compute digit precision of floating-point results
  \end{itemize}

\end{frame}

\begin{frame}
\frametitle{Digit Precision: Initial Evaluation}

\begin{tikzpicture}[scale=0.8]
  \begin{axis}[mark options={scale=1.0},height=7cm,width=12cm,grid=both, grid style={black!10}, 
      legend style={at={(0.95,0.35)}},
      legend cell align={left},
                              %x post scale=2.0, y post scale=2.0,
                              xmode=log,xmin=10,xmax=1e6,
                              xtick={1,10,100,1000,1e4,1e5,1e6,1e7}, xticklabels={1, $10^1$, $10^2$, $10^3$, $10^4$, $10^5$, $10^6$, $10^7$},
                              ymode=normal,ymin=0, ymax=50,
                              ytick={0, 5, 10, 15, 20, 25, 30, 35, 40, 45, 49},
                              yticklabels={0.0, 5.0, 10.0, 15.0, 20.0, 25.0, 30.0, 35.0, 40.0, 45.0, $+\infty$},
                              xlabel={Number of variables $n$}, ylabel={Digit Precision}
            ]


    \input{data-formatted/original-mpf+vars}
%    \input{data-formatted/upos-mpf+vars}
%    \input{data-formatted/epos-mpf+vars}
%    \input{data-formatted/optimized-product}
    \legend{
      \scriptsize \textsf{MC2024, Original},
%      \scriptsize \textsf{MC2024, Uniform$+$},
%      \scriptsize \textsf{MC2024, Exponential$+$},
%      \scriptsize \textsf{Optimized Product}
    }
    \addplot[mark=none, dashed] coordinates{(10,49) (1e6, 49)};) 
    \input{data-formatted/original-mpf+vars}
%    \addplot[mark=none] coordinates{(1,36.69) (1e7,30.69)};
%    \node[right] at (axis cs: 15, 33.2) {\rotatebox{-2.8}{Precision Bound}};
%    \addplot[mark=none, color=darkred] coordinates{(1,30) (1e7,30)};
%    \node[right] at (axis cs: 15, 28.0) {Target Precision};
 \end{axis}
\end{tikzpicture}

\textbf{Observations}
\begin{itemize}
\item High precision, even with billions of operations
\item Trend: Linear decrease with $n$
\end{itemize}

\end{frame}

\begin{frame}
  \frametitle{Deeper Evaluation}

  \textbf{CNF Benchmark Weights}
  \begin{itemize}
  \item Probabilities
    \begin{itemize}
      \bitem $w(x), w(\obar{x}) \;>\; 0$
      \bitem $w(x) + w(\obar{x})\; = \;1$
    \end{itemize}
  \item Unit weights
    \begin{itemize}
      \bitem $w(x) = w(\obar{x}) = 1$
    \end{itemize}
  \end{itemize}

\medskip

  \textbf{Added Weight Collections}

 \begin{center}
   \begin{tabular}{cccc}
%     \toprule
     Name & Range & Distribution & $w(x)\,:\,w(\obar{x})$ \\
     \midrule
     Uniform$+$      & $(0.0, 1.0)$         & Uniform     & $w(x) + w(\obar{x}) = 1$ \\
     Exponential$+$ & $[10^{-9}, 10^{+9}]$ & Exponential & Independent \\
%     \bottomrule
   \end{tabular}
 \end{center}

  \begin{itemize}
    \item Five randomly generated weight assignments for each formula
    \item Record lowest digit precision achieved
  \end{itemize}

\end{frame}


\begin{frame}
\frametitle{Digit Precision: Nonzero Weights}

\begin{tikzpicture}[scale=0.8]
  \begin{axis}[mark options={scale=1.0},height=7cm,width=12cm,grid=both, grid style={black!10}, 
      legend style={at={(0.95,0.35)}},
      legend cell align={left},
                              %x post scale=2.0, y post scale=2.0,
                              xmode=log,xmin=10,xmax=1e6,
                              xtick={1,10,100,1000,1e4,1e5,1e6,1e7}, xticklabels={1, $10^1$, $10^2$, $10^3$, $10^4$, $10^5$, $10^6$, $10^7$},
                              ymode=normal,ymin=0, ymax=50,
                              ytick={0, 5, 10, 15, 20, 25, 30, 35, 40, 45, 49},
                              yticklabels={0.0, 5.0, 10.0, 15.0, 20.0, 25.0, 30.0, 35.0, 40.0, 45.0, $+\infty$},
                              xlabel={Number of variables $n$}, ylabel={Digit Precision}
            ]


    \input{data-formatted/original-mpf+vars}
    \input{data-formatted/upos-mpf+vars}
    \input{data-formatted/epos-mpf+vars}
%    \input{data-formatted/optimized-product}
    \legend{
      \scriptsize \textsf{MC2024, Original},
      \scriptsize \textsf{MC2024, Uniform$+$},
      \scriptsize \textsf{MC2024, Exponential$+$},
%      \scriptsize \textsf{Optimized Product}
    }
    \addplot[mark=none, dashed] coordinates{(10,49) (1e6, 49)};) 
    \input{data-formatted/original-mpf+vars}
    \input{data-formatted/upos-mpf+vars}
    \input{data-formatted/epos-mpf+vars}
\only<2->{
    \addplot[mark=none] coordinates{(1,36.69) (1e7,30.69)};
    \node[right] at (axis cs: 15, 33.2) {\rotatebox{-2.8}{Precision Bound}};
}
%    \addplot[mark=none, color=darkred] coordinates{(1,30) (1e7,30)};
%    \node[right] at (axis cs: 15, 28.0) {Target Precision};
 \end{axis}
\end{tikzpicture}

\end{frame}

\end{document}
