\documentclass[t,pdf]{beamer}
\mode<presentation>{}


\usecolortheme[RGB={196, 30, 58}]{structure}


\usepackage{color}
\usepackage{animate}
\usepackage{tikz}
\usetikzlibrary{shadings,shadows}
\usetikzlibrary{shapes, arrows}
\usetikzlibrary{decorations.pathreplacing,angles,quotes}
\usetikzlibrary{calc}
\usetikzlibrary{positioning}
\usepackage{pgfplots}
\usepackage{graphicx}
\usepackage{adjustbox}
\usepackage{scrextend}
\usepackage{booktabs}


\usepackage{hyperref}
%% Colored hyperlink 
\newcommand{\cref}[2]{\href{#1}{\color{blue}#2}}
%% Colored hyperlink showing link in TT font
% \newcommand{\chref}[1]{\href{#1}{\small\tt \color{blue}#1}}
\newcommand{\hcref}[1]{\cref{#1}{\small\tt #1}}

\newcommand{\obar}[1]{\overline{#1}}
\newcommand{\nil}{\bot}
\newcommand{\bitem}{\item[$\bullet$]}
\newcommand{\opos}[1]{#1}
\newcommand{\oneg}[1]{\overline{#1}}
\newcommand{\xnot}{\oneg{x}}
\newcommand{\lit}{\ell}
\newcommand{\approximate}[1]{\hat{#1}}
\newcommand{\approxv}{\approximate{v}}
\newcommand{\approxw}{\approximate{w}}
\newcommand{\assign}{\alpha}
\newcommand{\modelset}{{\cal M}}
\newcommand{\aerror}{\delta}
\newcommand{\decimalprecision}{\Delta}
\newcommand{\vmin}{v^{-}}
\newcommand{\vmax}{v^{+}}
\newcommand{\interval}[1]{[\![#1]\!]}


\newcommand{\inputformula}{\phi}
\newcommand{\compiledformula}{\psi}

\usepackage{booktabs,colortbl}

\definecolor{redorange}{rgb}{0.878431, 0.235294, 0.192157}
\definecolor{lightblue}{rgb}{0.552941, 0.72549, 0.792157}
\definecolor{clearyellow}{rgb}{0.964706, 0.745098, 0}
\definecolor{clearorange}{rgb}{0.917647, 0.462745, 0}
\definecolor{mildgray}{rgb}{0.54902, 0.509804, 0.47451}
\definecolor{softblue}{rgb}{0.643137, 0.858824, 0.909804}
\definecolor{bluegray}{rgb}{0.141176, 0.313725, 0.603922}
\definecolor{lightgreen}{rgb}{0.709804, 0.741176, 0}
\definecolor{redpurple}{rgb}{0.835294, 0, 0.196078}
\definecolor{midblue}{rgb}{0, 0.592157, 0.662745}
\definecolor{clearpurple}{rgb}{0.67451, 0.0784314, 0.352941}
\definecolor{browngreen}{rgb}{0.333333, 0.313725, 0.145098}
\definecolor{darkestpurple}{rgb}{0.396078, 0.113725, 0.196078}
\definecolor{greypurple}{rgb}{0.294118, 0.219608, 0.298039}
\definecolor{darktruqoise}{rgb}{0, 0.239216, 0.298039}
\definecolor{darkbrown}{rgb}{0.305882, 0.211765, 0.160784}
\definecolor{midgreen}{rgb}{0.560784, 0.6, 0.243137}
\definecolor{darkred}{rgb}{0.576471, 0.152941, 0.172549}
\definecolor{darkpurple}{rgb}{0.313725, 0.027451, 0.470588}
\definecolor{darkestblue}{rgb}{0, 0.156863, 0.333333}
\definecolor{lightpurple}{rgb}{0.776471, 0.690196, 0.737255}
\definecolor{softgreen}{rgb}{0.733333, 0.772549, 0.572549}
\definecolor{offwhite}{rgb}{0.839216, 0.823529, 0.768627}

\definecolor{mediumgreen}{RGB}{20,140,20}
\definecolor{mediumblue}{RGB}{20,20,140}
\definecolor{medgreen}{rgb}{0.34, 0.65, 0.34}

\definecolor{dbl}{RGB}{49,97,160}
\definecolor{mpflow}{RGB}{96,165,200}
\definecolor{mpfmed}{RGB}{161,207,223}
\definecolor{mpfhigh}{RGB}{224,243,248}
\definecolor{mpfilow}{RGB}{215,48,39}
\definecolor{mpfimed}{RGB}{253,174,97}
\definecolor{mpfihigh}{RGB}{254,224,144}
\definecolor{mpq}{RGB}{100,100,100}


\newcommand{\red}[1]{\textcolor{red}{#1}}
\definecolor{darkred}{RGB}{180,0,0}
\newcommand{\darkred}[1]{\textcolor{darkred}{#1}}
\definecolor{dominocolor}{RGB}{0,0,128}
\definecolor{darkgreen}{RGB}{0,180,0}
\definecolor{darkgray}{RGB}{128,128,128}
%%\newcommand{\ground}{blue}

%%\newcommand{\ft}[1]{\frametitle{#1}}
%%\newcommand{\ig}[2]{\includegraphics[#1]{#2}}

\definecolor{xred}{rgb}{0.77, 0.12, 0.23}
\definecolor{xgreen}{rgb}{0.3, 0.6, 0}
\definecolor{xblue}{rgb}{0., 0.25, 1}

\newcommand{\btext}[1]{\textcolor{xblue}{#1}}
\newcommand{\rtext}[1]{\textcolor{xred}{#1}}
\newcommand{\gtext}[1]{\textcolor{xgreen}{#1}}
\newcommand{\wtext}[1]{\textcolor{white}{#1}}

\title{\huge Numerical Considerations \\ for Weighted Model Counting}
%\subtitle{}
\author{Randal E. Bryant}
\institute{\includegraphics[height=50pt]{CMU_Logo}}

\date{\textcolor{black}{Workshop on Counting, Sampling, and Synthesis 2025}}


\setbeamertemplate{footline}
{
	\leavevmode%
	\hbox{%
	\begin{beamercolorbox}[wd=0.35\paperwidth,ht=2.25ex,dp=1ex,center]{author in head/foot}%
%%	\tiny {\url{http://www.cs.cmu.edu/~bryant}}
%%			\vspace{4pt}
	\end{beamercolorbox}%
	\begin{beamercolorbox}[wd=0.45\paperwidth,ht=2.25ex,dp=1ex,center]{author in head/foot}%
	\end{beamercolorbox}%
	\begin{beamercolorbox}[wd=0.2\paperwidth,ht=2.5ex,dp=1ex,right]{date in head/foot}%
		\structure{\scriptsize \insertframenumber{} / \inserttotalframenumber\hspace*{3ex}}
		\vspace{3pt}
	\end{beamercolorbox}}%
	\vskip0pt%
}

\beamertemplatenavigationsymbolsempty

\begin{document}

\begin{frame}
	\titlepage

\small\url{http://www.cs.cmu.edu/~bryant}
\end{frame}

\begin{frame}

  \frametitle{Weighted Model Counting Definition}

  \begin{center}
\begin{tikzpicture}[scale=0.035]
  \node [left] at (0,25) {$\inputformula$};
  \node [right] at (220,25) {$w(\inputformula)$};
  \node [left] at (0,-20){$w$};

  \draw[fill=structure] (60,0) rectangle (160,50);

%  \draw[line width=2pt] (80,25) [-latex] -- (140,25);
  \draw[line width=2pt] (0,25) [-latex] -- (60,25);
  \draw[line width=2pt] (160,25) [-latex] -- (220,25);
  \draw[line width=2pt] (0,-20) -- (110,-20) [-latex] -- (110,0);

% \node [left] at (10,55) {\begin{tabular}{c} Input\\Formula \end{tabular}};

  \node [left] at (10,52) {Input};
  \node [left] at (10,40) {Formula};

  \node [left] at (10, 7) {Weight};
  \node [left] at (10,-5) {Assignment};

  \node [right] at (210,52) {Weighted};
  \node [right] at (210,40) {Count};


  \node[white] at (110,37) {Weighted};
  \node[white] at (110,25) {Model};
  \node[white] at (110,13) {Counter};
\end{tikzpicture}

\end{center}

\vskip -10pt
  {\bf Given}

  \begin{itemize}
  \item CNF formula $\inputformula$ over variables $X$
    \begin{itemize}
      \bitem Set of satisfying assignments (\emph{models}) $\modelset(\inputformula)$
      \bitem Each model $\assign$ is set of literals 
    \end{itemize}
  \item Weight assignment $w(x), w(\obar{x}) \in \mathbb{Q}$ for each $x \in X$
  \end{itemize}

  {\bf Compute}
  \begin{displaymath}
  w(\inputformula) \;\; \doteq \;\; \sum_{\assign \in \modelset(\inputformula)} \;\;\prod_{\lit \in \assign} w(\lit) 
  \end{displaymath}

\end{frame}

\begin{frame}
  \frametitle{Weighted Model Counting Computation}

\begin{center}
  \Large \emph{What numerical representation should be used?}
\end{center}

\bigskip
\begin{minipage}[t]{0.48\textwidth}
{\bf Rational Arithmetic}
\begin{itemize}
  \bitem Represent $v \in \mathbb{Q}$ as $v = a/b$
  \bitem Represent $a$ and $b$ as multiprecision integers
  \bitem Compute exact result
  \bitem High time and space cost
\end{itemize}
\end{minipage}
\begin{minipage}[t]{0.48\textwidth}
{\bf Floating-Point Arithmetic}
\begin{itemize}
  \bitem Approximate $v \in \mathbb{Q}$ as $\approxv = (-1)^s \times f \times 2^{e}$
  \bitem Represent fraction $f$ and exponent $e$ with fixed number of bits
  \bitem Low time and space cost
\end{itemize}
\end{minipage}

\bigskip
\only<2>{
\textbf{Assertion:}

\begin{itemize}
\item  Floating-point preferred
\item But only if desired precision can be guaranteed
\end{itemize}
}

\end{frame}


\begin{frame}
  \frametitle{Approximation Metrics}

For value $v$ and its approximation $\approxv$:

  \begin{displaymath}
\aerror[\approxv, v] \;\; \doteq \;\; \left\{ \begin{array}{lll}
  \frac{|\approxv - v|}{|v|}  & v \not = 0 & \textsf{Relative} \; \textsf{error}\\[0.5em]
  0 & v  = \approxv = 0 & 0 \approx 0\\[0.5em]
  1 & v = 0 \; \textsf{and} \; \approxv \not = 0 & 0 \not \approx \epsilon,\;\textsf{for} \;\epsilon \not = 0
  \end{array} \right.
\end{displaymath}

\medskip

\only<2>{
\textbf{Decimal Precision}

\begin{displaymath}
  \decimalprecision(\approxv, v) \;\; \doteq \;\; \max(0, -\log_{10} \aerror[\approxv, v])
\end{displaymath}

\begin{itemize}
 \item Number of trustworthy digits in decimal representation
\item  $= \infty$ when $v = \approxv$
\item $= 0$ when $v \not = 0$ and $\approxv = 0$
\item $= 0$ when $v = 0$ and $\approxv \not = 0$
\end{itemize}
} % only
\end{frame}

\begin{frame}

  \frametitle{Weighted Model Counting: Floating-Point Approximation}


\begin{tikzpicture}[scale=0.035]
  \node [left] at (0,75) {$D$};

  \node [left] at (0,25) {$\inputformula$};
  \node [right] at (220,25) {$\approxw(\inputformula)$};
  \node [right] at (220,11) {$\approx w(\inputformula)$};
  \node [left] at (0,-20){$w$};

  \draw[fill=structure] (60,0) rectangle (160,50);

%  \draw[line width=2pt] (80,25) [-latex] -- (140,25);
  \draw[line width=2pt] (0,75) -- (110,75) [-latex] -- (110,50);

  \draw[line width=2pt] (0,25) [-latex] -- (60,25);
  \draw[line width=2pt] (160,25) [-latex] -- (220,25);
  \draw[line width=2pt] (0,-20) -- (110,-20) [-latex] -- (110,0);

% \node [left] at (10,55) {\begin{tabular}{c} Input\\Formula \end{tabular}};

  \node [left] at (10,102) {Target};
  \node [left] at (10,90) {Precision};


  \node [left] at (10,52) {Input};
  \node [left] at (10,40) {Formula};

  \node [left] at (10, 7) {Weight};
  \node [left] at (10,-5) {Assignment};

  \node [right] at (210,64) {Approximate};
  \node [right] at (210,52) {Weighted};
  \node [right] at (210,40) {Count};


  \node[white] at (110,37) {Weighted};
  \node[white] at (110,25) {Model};
  \node[white] at (110,13) {Counter};
\end{tikzpicture}

\bigskip

  {\bf Approximation}

  \begin{itemize}
  \item Set target precision $D$
  \item Compute approximation $\approxw(\inputformula)$
  \item Require $\decimalprecision(\approxw, w) \geq D$
  \end{itemize}

\end{frame}

\begin{frame}
  \frametitle{Choosing Target Precision}

\bigskip
  $D = 1$
  \begin{itemize}
    \item $10\%$ accuracy
    \item Good enough for 2020 Weighted Model Competition

  \end{itemize}

\bigskip
  $D = 3$
  \begin{itemize}
    \item $0.1\%$ accuracy
    \item Good enough for 2024 Weighted Model Competition
  \end{itemize}

\bigskip
\only<2>{
  $D = 30$
  \begin{itemize}
  \item Distance from Earth to Alpha Centauri $\approx 4.37$ light years.
  \item $D=30$ can represent distance from Earth to Alpha Centauri to nearest picometer
  \item Hydrogen atom diameter $\approx 106$ picometers.
  \end{itemize}
}

\end{frame}

\begin{frame}

\frametitle{Using Knowledge Compilation}

  \begin{center}
\begin{tikzpicture}[scale=0.035]
  \node [left] at (0,75) {$D$};
  \node [left] at (0,25) {$\inputformula$};
  \node [right] at (220,25) {$\approxw(\psi)$};
  \node [right] at (220,11) {$\approx w(\inputformula)$};
  \node [left] at (0,-20){$w$};

  \node at (110,25) {$\psi$};

  \draw[fill=structure] (20,0) rectangle (80,50);
  \draw[fill=structure] (140,0) rectangle (200,50);

  \draw[line width=2pt] (0,25) [-latex] -- (20,25);
%  \draw[line width=2pt] (80,25) [-latex] -- (140,25);

  \draw[line width=2pt] (80,25) [-latex] -- (100,25);  
  \draw[line width=2pt] (120,25) [-latex] -- (140,25);

  \draw[line width=2pt] (200,25) [-latex] -- (220,25);

  \draw[line width=2pt] (0,75) -- (170,75) [-latex] -- (170,50);
  \draw[line width=2pt] (0,-20) -- (170,-20) [-latex] -- (170,0);

% \node [left] at (10,55) {\begin{tabular}{c} Input\\Formula \end{tabular}};
  \node [left] at (10,102) {Target};
  \node [left] at (10,90) {Precision};


  \node [left] at (10,52) {Input};
  \node [left] at (10,40) {Formula};

  \node [left] at (10, 7) {Weight};
  \node [left] at (10,-5) {Assignment};

  
  \node at (110,60) {decision-DNNF};
  \node at (110,48) {Formula};

  \node [right] at (210,64) {Approximate};
  \node [right] at (210,52) {Weighted};
  \node [right] at (210,40) {Count};


  \node[white] at (50,31) {Knowledge};
  \node[white] at (50,19) {Compiler};

  \node[white] at (170,37) {Weighted};
  \node[white] at (170,25) {Model};
  \node[white] at (170,13) {Counter};
\end{tikzpicture}
\end{center}

  \begin{itemize}
  \item Convert input formula $\inputformula$ into logically equivalent formula $\psi$
  \item $\psi$ has structure that makes weighted counting tractable
  \end{itemize}

\end{frame}


\begin{frame}
  \frametitle{Experimental Exploration}

\medskip  

  \textbf{Benchmark Formulas}
  \begin{itemize}
  \item 200 formulas from 2024 Weighted Model Competition
  \item Compile with D4 version 2 knowledge compiler
    \begin{itemize}
    \bitem Compiled 100 within limits of 3600 seconds and 64~GB
%    \bitem 33 -- 325,113 variables
%    \bitem 209 -- $1.66\times 10^9$ binary operations in compiled representation
    \bitem Up to 325,113 variables
    \bitem Up to $1.66\times 10^9$ binary operations in compiled representation
    \end{itemize}
  \end{itemize}

\medskip

    \textbf{Counting}
    \begin{itemize}
    \item Use weights included in CNF file
    \item Exact result using GMP~MPQ rational arithmetic library
      \begin{itemize}
       \bitem Memory limit exceeded for two formulas
      \end{itemize}
    \item Floating-point using GMP~MPF  library
      \begin{itemize}
       \bitem Fraction size $p = 128$
      \end{itemize}
    \end{itemize}

\medskip

  \textbf{Evaluation}
  \begin{itemize}
  \item Compute decimal precision of floating-point results
  \end{itemize}

\end{frame}

\begin{frame}
\frametitle{Decimal Precision: Initial Evaluation}

\medskip

\begin{tikzpicture}[scale=0.8]
  \begin{axis}[mark options={scale=1.0},height=7cm,width=12cm,grid=both, grid style={black!10}, 
      legend style={at={(0.95,0.35)}},
      legend cell align={left},
                              %x post scale=2.0, y post scale=2.0,
                              xmode=log,xmin=10,xmax=1e6,
                              xtick={1,10,100,1000,1e4,1e5,1e6,1e7}, xticklabels={1, $10^1$, $10^2$, $10^3$, $10^4$, $10^5$, $10^6$, $10^7$},
                              ymode=normal,ymin=0, ymax=50,
                              ytick={0, 5, 10, 15, 20, 25, 30, 35, 40, 45, 49},
                              yticklabels={0.0, 5.0, 10.0, 15.0, 20.0, 25.0, 30.0, 35.0, 40.0, 45.0, $+\infty$},
                              xlabel={Number of variables $n$}, ylabel={Decimal Precision}
            ]


    \input{data-formatted/original-mpf+vars}
%    \input{data-formatted/upos-mpf+vars}
%    \input{data-formatted/epos-mpf+vars}
%    \input{data-formatted/optimized-product}
    \legend{
      \scriptsize \textsf{MC2024, Original},
%      \scriptsize \textsf{MC2024, Uniform$+$},
%      \scriptsize \textsf{MC2024, Exponential$+$},
%      \scriptsize \textsf{Optimized Product}
    }
    \addplot[mark=none, dashed] coordinates{(10,49) (1e6, 49)};) 
    \input{data-formatted/original-mpf+vars}
%    \addplot[mark=none] coordinates{(1,36.69) (1e7,30.69)};
%    \node[right] at (axis cs: 15, 33.2) {\rotatebox{-2.8}{Precision Bound}};
%    \addplot[mark=none, color=darkred] coordinates{(1,30) (1e7,30)};
%    \node[right] at (axis cs: 15, 28.0) {Target Precision};
 \end{axis}
\end{tikzpicture}

\textbf{Observations}
\begin{itemize}
\item High precision, even with billions of operations
\item Trend: Lose one digit of precision when increase   $n$  by $10\times$.
\end{itemize}

\end{frame}

\begin{frame}
  \frametitle{Enlarging Benchark Set}

%   \textbf{CNF Benchmark Weights}
%   \begin{itemize}
%   \item Probabilities
%     \begin{itemize}
%       \bitem $w(x), w(\obar{x}) \;>\; 0$
%       \bitem $w(x) + w(\obar{x})\; = \;1$
%     \end{itemize}
%   \item Unit weights
%     \begin{itemize}
%       \bitem $w(x) = w(\obar{x}) = 1$
%     \end{itemize}
%   \end{itemize}
% 
% \medskip

\bigskip

  \textbf{Nonnegative Weight Collections}

 \begin{center}
   \begin{tabular}{cccc}
%     \toprule
     Name & Range & Distribution & $w(x)\,:\,w(\obar{x})$ \\
     \midrule
     Uniform$+$      & $(0.0, 1.0)$         & Uniform     & $w(x) + w(\obar{x}) = 1$ \\[0.5em]
     Exponential$+$ & $[10^{-9}, 10^{+9}]$ & Exponential & Independent \\
%     \bottomrule
   \end{tabular}
 \end{center}

\bigskip

\textbf{Collection:}

  \begin{itemize}
    \item Five randomly generated weight assignments for each formula
    \item Record lowest decimal precision achieved
  \end{itemize}

\end{frame}


\begin{frame}
\frametitle{Decimal Precision: Nonnegative Weights}

\medskip

\begin{tikzpicture}[scale=0.8]
  \begin{axis}[mark options={scale=1.0},height=7cm,width=12cm,grid=both, grid style={black!10}, 
      legend style={at={(0.95,0.35)}},
      legend cell align={left},
                              %x post scale=2.0, y post scale=2.0,
                              xmode=log,xmin=10,xmax=1e6,
                              xtick={1,10,100,1000,1e4,1e5,1e6,1e7}, xticklabels={1, $10^1$, $10^2$, $10^3$, $10^4$, $10^5$, $10^6$, $10^7$},
                              ymode=normal,ymin=0, ymax=50,
                              ytick={0, 5, 10, 15, 20, 25, 30, 35, 40, 45, 49},
                              yticklabels={0.0, 5.0, 10.0, 15.0, 20.0, 25.0, 30.0, 35.0, 40.0, 45.0, $+\infty$},
                              xlabel={Number of variables $n$}, ylabel={Decimal Precision}
            ]


    \input{data-formatted/original-mpf+vars}
\only<2->{    \input{data-formatted/upos-mpf+vars}}
\only<3->{    \input{data-formatted/epos-mpf+vars}}
%    \input{data-formatted/optimized-product}
\only<1>{\legend{\scriptsize \textsf{MC2024, Original}}}
\only<2>{\legend{\scriptsize \textsf{MC2024, Original}, \scriptsize \textsf{MC2024, Uniform$+$}}}
\only<3->{\legend{\scriptsize \textsf{MC2024, Original},
                   \scriptsize \textsf{MC2024, Uniform$+$},
                   \scriptsize \textsf{MC2024, Exponential$+$}}}
      
    \addplot[mark=none, dashed] coordinates{(10,49) (1e6, 49)};) 
    \input{data-formatted/original-mpf+vars}
\only<2->{    \input{data-formatted/upos-mpf+vars}}
\only<3->{    \input{data-formatted/epos-mpf+vars}}
\only<4->{
    \addplot[mark=none] coordinates{(1,36.69) (1e7,30.69)};
    \node[right] at (axis cs: 15, 33.2) {\rotatebox{-2.8}{Precision Bound}};
}
%    \addplot[mark=none, color=darkred] coordinates{(1,30) (1e7,30)};
%    \node[right] at (axis cs: 15, 28.0) {Target Precision};
 \end{axis}
\end{tikzpicture}

\only<2->{
\begin{itemize}
\item Precision stays high
\item Same trend holds
\only<4->{\item \emph{Theorem}: Precision guaranteed for nonnegative weights}
\end{itemize}
}
\end{frame}

\begin{frame}
  \frametitle{Precision Bound Theorem}

\bigskip

  \textbf{Assuming:}
  \begin{itemize}
  \item decision-DNNF formula $\psi$ over $n$ variables
  \item Nonnegative weights: $w(x), w(\obar{x}) \geq 0$ for all $x \in X$
  \item Floating point arithmetic with fraction size $p$
  \item Parameters $n$ and $p$ satisfy $\log_2 n \;\leq\; p/2-1$
  \item No overflow or underflow
  \end{itemize}

\medskip

\textbf{Guarantee:}
\begin{displaymath}
\decimalprecision(\approxw(\psi), w(\psi)) \;\; \geq \;\; p \cdot \log_{10}2 \rtext{- \log_{10}n} - \log_{10} 7
\end{displaymath}

\textbf{Key Properties:}
\begin{itemize}
\item Independent of weight values
\item Independent of formula size
\end{itemize}
\end{frame}

\begin{frame}
\frametitle{Application of Precision Bound Theorem}

\begin{itemize}
\item Problem with $n$ variables
\item All weights nonnegative
\item Want to guarantee $\decimalprecision(\approxw, w) \geq D$
\item Choose fraction size $p$ such that two conditions hold:
\begin{displaymath}
\begin{array}{rcl}
  p & \geq & 2(1 + \log_2 n)  \\[0.5em]
  p & \geq & D \cdot \log_2 10 +\log_2 n + 2.9 
\end{array}
\end{displaymath}
\end{itemize}

\smallskip

{\bf For $n = 10^6$:}

\begin{itemize}
\item First condition: $p \geq 44.8$
\item Second condition:
  \begin{center}
    \begin{tabular}{rrl}
      \makebox[40pt]{$D$} & \makebox[50pt]{Min.~$p$} & Comment\\
      \midrule
      $1$ & $26.2$ & Dominated by first condition\\
      $9$ & $52.8$ & IEEE Double \\
      $31$ & $125.9$ & MPF-128 \\
    \end{tabular}
  \end{center}
\end{itemize}
\end{frame}

\begin{frame}

\frametitle{Evaluating Nonnegative Benchmarks}

\bigskip

\begin{tikzpicture}[scale = 0.80]
  \begin{axis}[
      ybar stacked,
      width = 12cm,
      height=4.5cm,
      bar width=15pt,
%%      nodes near coords,
%%      enlargelimits=0.15,
      legend style={at={(0.435,-0.70)},
      anchor=south, legend columns=-1},
      ylabel = {Instance count},
      xlabel = {Target precision $D$},
      ytick = {500, 1000, 1500, 2000, 2500},
      yticklabels ={500, 1000, 1500, 2000, 2500},
      symbolic x coords={1, 5, 10, 15, 20, 25, 30, 35, 40, 45, 50, 55, 60, 65, 70},
      xtick=data,
      ]
\input{data-formatted/tabulate-count-nonneg}
\legend{\strut Double, \strut MPF-64, \strut MPF-128, \strut MPF-256}
  \end{axis}
\end{tikzpicture}

\medskip

\begin{itemize}
\item 100 formulas $\times$ 2 weight collections $\times$ 5 random assignments
\item When Double overflows or underflows, recompute with MPF-64
\item Rely on Precision Bound Theorem to guarantee precision
\item No redundant work required
\end{itemize}

\end{frame}


\begin{frame}
  \frametitle{What About Mixed Positive/Negative Weights?}

\medskip

  \textbf{Mixed Weight Collections}

\begin{center}
   \begin{tabular}{ccccc}
%     \toprule
     Name & Magnitude  & Dist. & Sign & $w(x)\,:\,w(\obar{x})$ \\
     \midrule
     Uniform$\pm$      & $(0.0, 1.0)$     & Uni.  & Random $\pm$   & Ind. \\[0.5em]
     Exponential$\pm$ & $[10^{-9}, 10^{+9}]$ & Exp. & Random $\pm$ & Ind. \\
%     \bottomrule
   \end{tabular}
\end{center}

\bigskip

\textbf{Collection:}

  \begin{itemize}
    \item Five randomly generated weight assignments for each formula
    \item Record lowest decimal precision achieved
  \end{itemize}

\end{frame}

\begin{frame}
\frametitle{Decimal Precision: Mixed Weights}

\medskip

\begin{tikzpicture}[scale=0.8]
  \begin{axis}[mark options={scale=1.0},height=7cm,width=12cm,grid=both, grid style={black!10}, 
      legend style={at={(0.95,0.35)}},
      legend cell align={left},
                              %x post scale=2.0, y post scale=2.0,
                              xmode=log,xmin=10,xmax=1e6,
                              xtick={1,10,100,1000,1e4,1e5,1e6,1e7}, xticklabels={1, $10^1$, $10^2$, $10^3$, $10^4$, $10^5$, $10^6$, $10^7$},
                              ymode=normal,ymin=0, ymax=45,
                              ytick={0, 5, 10, 15, 20, 25, 30, 35, 40, 45},
                              yticklabels={0.0, 5.0, 10.0, 15.0, 20.0, 25.0, 30.0, 35.0, 40.0, 45.0},
                              xlabel={Number of variables $n$}, ylabel={Decimal Precision}
            ]



    \input{data-formatted/uposneg-mpf+vars}
    \input{data-formatted/eposneg-mpf+vars}
    \legend{
      \scriptsize \textsf{MC2024, Uniform$\pm$},
      \scriptsize \textsf{MC2024, Exponential$\pm$},
    }
    \addplot[mark=none, dashed] coordinates{(10,49) (1e6, 49)};) 
    \addplot[mark=none, dashed] coordinates{(1,36.69) (1e7,30.69)};
    \node[right] at (axis cs: 15, 33.2) {\rotatebox{-2.8}{Precision Bound?}};
%    \addplot[mark=none, color=darkred] coordinates{(1,30) (1e7,30)};
%    \node[right] at (axis cs: 15, 28.0) {Target Precision};
 \end{axis}
\end{tikzpicture}

\textbf{Observations}
\begin{itemize}
\item Seems to be OK
\end{itemize}

\end{frame}



\begin{frame}
  \frametitle{What About Mixed Weights?}

\medskip

  \textbf{Challenging Weight Collection}

\medskip

  \begin{itemize}
  \item
    Weight assignment with high dynamic range and homogeneity
  \end{itemize}

\medskip

\begin{center}
   \begin{tabular}{ccccc}
%     \toprule
     Name & Magnitude  & Dist. & Sign & $w(x)\,:\,w(\obar{x})$ \\
     \midrule
     Uniform$\pm$      & $(0.0, 1.0)$     & Uni.  & Rand.~$\pm$   & Ind. \\[0.5em]
     Exponential$\pm$ & $[10^{-9}, 10^{+9}]$ & Exp. & Rand.~$\pm$ & Ind. \\[0.5em]
     \rtext{Limits$\pm$} & \rtext{$\{10^{-9}, 10^{+9}\}$} & \rtext{Rand.} & \rtext{Rand.~$\pm$} & \rtext{$w(x) + w(\obar{x}) \not = 0$} \\
%     \bottomrule
   \end{tabular}
\end{center}

\medskip

%%  \begin{itemize}
%%    \item Five randomly generated weight assignments for each formula
%%    \item Record lowest decimal precision achieved
%%  \end{itemize}

\end{frame}

\begin{frame}
\frametitle{Decimal Precision: Challenging Weight Assignments}

\medskip

\begin{tikzpicture}[scale=0.8]
  \begin{axis}[mark options={scale=1.0},height=7cm,width=12cm,grid=both, grid style={black!10}, 
      legend style={at={(0.95,0.35)}},
      legend cell align={left},
                              %x post scale=2.0, y post scale=2.0,
                              xmode=log,xmin=10,xmax=1e6,
                              xtick={1,10,100,1000,1e4,1e5,1e6,1e7}, xticklabels={1, $10^1$, $10^2$, $10^3$, $10^4$, $10^5$, $10^6$, $10^7$},
                              ymode=normal,ymin=0, ymax=45,
                              ytick={0, 5, 10, 15, 20, 25, 30, 35, 40, 45},
                              yticklabels={0.0, 5.0, 10.0, 15.0, 20.0, 25.0, 30.0, 35.0, 40.0, 45.0},
                              xlabel={Number of variables $n$}, ylabel={Decimal Precision}
            ]



    \input{data-formatted/uposneg-mpf+vars}
    \input{data-formatted/eposneg-mpf+vars}
    \input{data-formatted/bposneg-mpf+vars}
    \legend{
      \scriptsize \textsf{MC2024, Uniform$\pm$},
      \scriptsize \textsf{MC2024, Exponential$\pm$},
      \scriptsize \textsf{MC2024, Limits$\pm$},
    }
    \addplot[mark=none, dashed] coordinates{(10,49) (1e6, 49)};) 
    \addplot[mark=none, dashed] coordinates{(1,36.69) (1e7,30.69)};
    \node[right] at (axis cs: 15, 33.2) {\rotatebox{-2.8}{Invalid Precision Bound}};
%    \addplot[mark=none, color=darkred] coordinates{(1,30) (1e7,30)};
%    \node[right] at (axis cs: 15, 28.0) {Target Precision};
 \end{axis}
\end{tikzpicture}

\textbf{Observations}
\begin{itemize}
\item Many failures
\item Interesting band structure
\end{itemize}

\end{frame}

\begin{frame}

\frametitle{Interval Arithmetic}

\medskip

\textbf{MPFI Software Library}
\begin{itemize}
\item Maintain values of form $\interval{\vmin, \vmax}$
\item $\vmin$ and $\vmax$ represented in floating-point (MPF)
\item Arithmetic operations generalized to operate on intervals
\item Guarantee approximation of true value $v^{-} \leq v \leq v^{+}$
\end{itemize}

\bigskip

\textbf{Interval Decimal Precision}

\begin{displaymath}
\aerror(\interval{\vmin, \vmax}) \;\;\; = \;\;\; \left\{ \begin{array}{ll}
  \frac{\vmax - \vmin}{\min(|\vmin|, |\vmax|)}  & 0 \not \in \interval{\vmin, \vmax}\\[0.8em]
  0 & \vmin = \vmax = 0 \\
  1 & 0 \in \interval{\vmin, \vmax} \;\; \textrm{and} \;\; \vmin < \vmax
  \end{array} \right. \label{eqn:interval:error}
\end{displaymath}

\begin{itemize}
  \item Use to estimate precision of computed value $\approxw(\psi)$
\end{itemize}

\end{frame}



\begin{frame}

\frametitle{Evaluating Mixed-Sign  Benchmarks with Hybrid Approach}

\medskip

\begin{tikzpicture}[scale = 0.80]
  \begin{axis}[
      ybar stacked,
      width = 12cm,
      height=6.0cm,
      bar width=15pt,
%%      nodes near coords,
%%      enlargelimits=0.15,
      legend style={at={(0.435,-0.50)},
      anchor=south, legend columns=-1},
      ylabel = {Instance count},
      xlabel = {Target precision $D$},
      ytick = {500, 1000, 1500, 2000, 2500},
      yticklabels ={500, 1000, 1500, 2000, 2500},
      symbolic x coords={1, 5, 10, 15, 20, 25, 30, 35, 40, 45, 50, 55, 60, 65, 70},
      xtick=data,
      ]
\input{data-formatted/tabulate-count-posneg}
\legend{MPFI-64, \strut MPFI-128, \strut MPFI-256, \strut MPQ}
  \end{axis}
\end{tikzpicture}

\smallskip

\begin{itemize}
\item 100 formulas $\times$ 3 weight collections $\times$ 5 random assignments
\item Use progressively larger MPFI fraction sizes until target achieved
\item Use MPQ when interval arithmetic fails
\item May require redundant work
\end{itemize}

\end{frame}

\begin{frame}

\frametitle{Evaluating All Benchmarks with Hybrid Approach}

\medskip

\begin{tikzpicture}[scale = 0.80]
  \begin{axis}[
      ybar stacked,
      ymax = 2500,
      width = 12cm,
      height=8cm,
      bar width=15pt,
%%      nodes near coords,
%%      enlargelimits=0.15,
      legend style={at={(0.435,-0.30)},
      anchor=south, legend columns=-1},
      ylabel = {Instance count},
      xlabel = {Target precision $D$},
      ytick = {500, 1000, 1500, 2000, 2500},
      yticklabels ={500, 1000, 1500, 2000, 2500},
      symbolic x coords={1, 5, 10, 15, 20, 25, 30, 35, 40, 45, 50, 55, 60, 65, 70},
      xtick=data,
      ]
\input{data-formatted/tabulate-count}
\legend{\strut Double, \strut MPF-64, \strut MPF-128, \strut MPF-256, \strut MPFI-64, \strut MPFI-128, \strut MPFI-256, \strut MPQ}
  \end{axis}
\end{tikzpicture}

\medskip

\begin{itemize}
\item 100 formulas $\times$ 5 weight collections $\times$ 5 random assignments
\item 1000 nonegative weights + 1500 mixed-sign weights
\end{itemize}

\end{frame}


\begin{frame}

\medskip

\frametitle{Hybrid Benchmark Cumulative Evaluation Time}

\begin{tikzpicture}[scale = 0.80]
  \begin{axis}[
      ybar stacked,
      width = 12cm,
      height=8cm,
      bar width=15pt,
%%      nodes near coords,
%%      enlargelimits=0.15,
      legend style={at={(0.42,-0.30)},
        anchor=north, legend columns=-1},
      ylabel = {Total Time (hours)},
      xlabel = {Target precision $D$},
      ymax = 30,
      ytick = {5, 10, 15, 20, 25, 30},
      yticklabels = {5, 10, 15, 20, 25, 30},
      symbolic x coords={1, 5, 10, 15, 20, 25, 30, 35, 40, 45, 50, 55, 60, 65, 70},
      xtick=data,
      ]
\input{data-formatted/tabulate-effort}
\legend{\strut Double, \strut MPF-64, \strut MPF-128, \strut MPF-256, \strut MPFI-64, \strut MPFI-128, \strut MPFI-256, \strut MPQ}
  \end{axis}
\end{tikzpicture}
  
\end{frame}

\begin{frame}
  \frametitle{Conclusions}

  \bigskip
  
\textbf{Observations}

\medskip

  \begin{itemize}
  \item Can reliably use floating-point for nonnegative weights
  \item Hybrid interval + rational works well for mixed weights
  \item Scheme works across wide range of target precisions
  \end{itemize}

  \bigskip
  
\textbf{Generalizations}  

  \begin{itemize}
  \item Ideas extend to other representations
  \item d-DNNF
  \item BDDs, MDDs, SDDs, etc.
  \end{itemize}

\end{frame}

\end{document}

\begin{frame}
  \frametitle{Floating-Point Cancellation}

\medskip
\textbf{IEEE Double Examples:}

\smallskip

  \begin{tabular}{rcl}
    $(10^{15} + 3.1416) - 10^{15}$ & $\longrightarrow$ & $3.125$ \\[0.5em]
%    $(10^{16} + 3.1416) - 10^{16}$ & $\longrightarrow$ & $4.000$ \\[0.5em]
    $(10^{20} + 3.1416) - 10^{20}$ & $\longrightarrow$ & $0.000$ \\[0.5em]
    $(10^{15} + 3.1416) - (10^{15} + 0.1)$ & $\longrightarrow$ & $3.000$ \\[0.5em]
  \end{tabular}

\medskip

\textbf{General Case: For $T \gg s$ and $T \approx T'$}

\medskip

  \begin{tabular}{rcl}
    $(T + s) - T$ & $\longrightarrow$ & $0$ \\[0.5em]
    $(T + s) - T'$ & $\longrightarrow$ & ?? \\
  \end{tabular}

  \medskip

  \textbf{Requirements for Cancellation}

\smallskip

  \begin{itemize}
  \item Wide dynamic range: $T \gg s$
    \begin{itemize}
      \bitem Not present in Uniform$\pm$
    \end{itemize}
  \item Homogeneity: $T \approx T'$
    \begin{itemize}
      \bitem Not present in Exponential$\pm$
    \end{itemize}

  \end{itemize}

\end{frame}

\begin{frame}

\frametitle{Hybrid Approach}

\smallskip

\begin{itemize}
\item decision-DNNF formula with $n$ variables
\item Want to guarantee decimal precision $D$
\end{itemize}

\medskip

\textbf{Nonnegative Weight Assignments}

\smallskip

\begin{itemize}
\item Compute required fraction size $p$
\item Use IEEE double or MPF with appropriate fraction size
  \begin{itemize}
    \bitem Test for overflow / underflow with double
    \bitem Switch to MPF when needed
  \end{itemize}
\item Theorem provides precision guarantee
\end{itemize}

\medskip

\textbf{Weight Assignments with Negative Numbers}

\smallskip

\begin{itemize}
\item Start with fraction size $p$ based on nonnegative case
\item Use MPFI with increasing fraction sizes
\item Use MPQ as last resort
\end{itemize}

\end{frame}
