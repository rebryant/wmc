\documentclass[t,pdf]{beamer}
\mode<presentation>{}


\usecolortheme[RGB={196, 30, 58}]{structure}


\usepackage{color}
\usepackage{animate}
\usepackage{tikz}
\usetikzlibrary{shadings,shadows}
\usetikzlibrary{shapes, arrows}
\usetikzlibrary{decorations.pathreplacing,angles,quotes}
\usetikzlibrary{calc}
\usetikzlibrary{positioning}
\usepackage{pgfplots}
\usepackage{graphicx}
\usepackage{adjustbox}
\usepackage{scrextend}
\usepackage{booktabs}


\usepackage{hyperref}
%% Colored hyperlink 
\newcommand{\cref}[2]{\href{#1}{\color{blue}#2}}
%% Colored hyperlink showing link in TT font
% \newcommand{\chref}[1]{\href{#1}{\small\tt \color{blue}#1}}
\newcommand{\hcref}[1]{\cref{#1}{\small\tt #1}}

\newcommand{\obar}[1]{\overline{#1}}
\newcommand{\nil}{\bot}
\newcommand{\bitem}{\item[$\bullet$]}
\newcommand{\opos}[1]{#1}
\newcommand{\oneg}[1]{\overline{#1}}
\newcommand{\xnot}{\oneg{x}}
\newcommand{\lit}{\ell}
\newcommand{\approximate}[1]{\hat{#1}}
\newcommand{\approxv}{\approximate{v}}
\newcommand{\approxw}{\approximate{w}}
\newcommand{\assign}{\alpha}
\newcommand{\modelset}{{\cal M}}
\newcommand{\aerror}{\delta}
\newcommand{\decimalprecision}{\Delta}
\newcommand{\vmin}{v^{-}}
\newcommand{\vmax}{v^{+}}
\newcommand{\interval}[1]{[\![#1]\!]}


\newcommand{\inputformula}{\phi}
\newcommand{\compiledformula}{\psi}

\usepackage{booktabs,colortbl}

\definecolor{redorange}{rgb}{0.878431, 0.235294, 0.192157}
\definecolor{lightblue}{rgb}{0.552941, 0.72549, 0.792157}
\definecolor{clearyellow}{rgb}{0.964706, 0.745098, 0}
\definecolor{clearorange}{rgb}{0.917647, 0.462745, 0}
\definecolor{mildgray}{rgb}{0.54902, 0.509804, 0.47451}
\definecolor{softblue}{rgb}{0.643137, 0.858824, 0.909804}
\definecolor{bluegray}{rgb}{0.141176, 0.313725, 0.603922}
\definecolor{lightgreen}{rgb}{0.709804, 0.741176, 0}
\definecolor{redpurple}{rgb}{0.835294, 0, 0.196078}
\definecolor{midblue}{rgb}{0, 0.592157, 0.662745}
\definecolor{clearpurple}{rgb}{0.67451, 0.0784314, 0.352941}
\definecolor{browngreen}{rgb}{0.333333, 0.313725, 0.145098}
\definecolor{darkestpurple}{rgb}{0.396078, 0.113725, 0.196078}
\definecolor{greypurple}{rgb}{0.294118, 0.219608, 0.298039}
\definecolor{darktruqoise}{rgb}{0, 0.239216, 0.298039}
\definecolor{darkbrown}{rgb}{0.305882, 0.211765, 0.160784}
\definecolor{midgreen}{rgb}{0.560784, 0.6, 0.243137}
\definecolor{darkred}{rgb}{0.576471, 0.152941, 0.172549}
\definecolor{darkpurple}{rgb}{0.313725, 0.027451, 0.470588}
\definecolor{darkestblue}{rgb}{0, 0.156863, 0.333333}
\definecolor{lightpurple}{rgb}{0.776471, 0.690196, 0.737255}
\definecolor{softgreen}{rgb}{0.733333, 0.772549, 0.572549}
\definecolor{offwhite}{rgb}{0.839216, 0.823529, 0.768627}

\definecolor{mediumgreen}{RGB}{20,140,20}
\definecolor{mediumblue}{RGB}{20,20,140}
\definecolor{medgreen}{rgb}{0.34, 0.65, 0.34}

\definecolor{dbl}{RGB}{49,97,160}
\definecolor{mpflow}{RGB}{96,165,200}
\definecolor{mpfmed}{RGB}{161,207,223}
\definecolor{mpfhigh}{RGB}{224,243,248}
\definecolor{mpfilow}{RGB}{215,48,39}
\definecolor{mpfimed}{RGB}{253,174,97}
\definecolor{mpfihigh}{RGB}{254,224,144}
\definecolor{mpq}{RGB}{100,100,100}


\newcommand{\red}[1]{\textcolor{red}{#1}}
\definecolor{darkred}{RGB}{180,0,0}
\newcommand{\darkred}[1]{\textcolor{darkred}{#1}}
\definecolor{dominocolor}{RGB}{0,0,128}
\definecolor{darkgreen}{RGB}{0,180,0}
\definecolor{darkgray}{RGB}{128,128,128}
%%\newcommand{\ground}{blue}

%%\newcommand{\ft}[1]{\frametitle{#1}}
%%\newcommand{\ig}[2]{\includegraphics[#1]{#2}}

\definecolor{xred}{rgb}{0.77, 0.12, 0.23}
\definecolor{xgreen}{rgb}{0.3, 0.6, 0}
\definecolor{xblue}{rgb}{0., 0.25, 1}

\newcommand{\btext}[1]{\textcolor{xblue}{#1}}
\newcommand{\rtext}[1]{\textcolor{xred}{#1}}
\newcommand{\gtext}[1]{\textcolor{xgreen}{#1}}
\newcommand{\wtext}[1]{\textcolor{white}{#1}}

\title{\huge Numerical Considerations \\ for Weighted Model Counting}
%\subtitle{}
\author{Randal E. Bryant}
\institute{\includegraphics[height=50pt]{CMU_Logo}}

\date{\textcolor{black}{Workshop on Counting, Sampling, and Synthesis 2025}}


\setbeamertemplate{footline}
{
	\leavevmode%
	\hbox{%
	\begin{beamercolorbox}[wd=0.35\paperwidth,ht=2.25ex,dp=1ex,center]{author in head/foot}%
%%	\tiny {\url{http://www.cs.cmu.edu/~bryant}}
%%			\vspace{4pt}
	\end{beamercolorbox}%
	\begin{beamercolorbox}[wd=0.45\paperwidth,ht=2.25ex,dp=1ex,center]{author in head/foot}%
	\end{beamercolorbox}%
	\begin{beamercolorbox}[wd=0.2\paperwidth,ht=2.5ex,dp=1ex,right]{date in head/foot}%
		\structure{\scriptsize \insertframenumber{} / \inserttotalframenumber\hspace*{3ex}}
		\vspace{3pt}
	\end{beamercolorbox}}%
	\vskip0pt%
}

\beamertemplatenavigationsymbolsempty

\begin{document}


\begin{frame}

  \frametitle{Weighted Model Counting Definition}

  \begin{center}
\begin{tikzpicture}[scale=0.035]
  \node [left] at (0,25) {$\inputformula$};
  \node [right] at (220,25) {$w(\inputformula)$};
  \node [left] at (0,-20){$w$};

  \draw[fill=structure] (60,0) rectangle (160,50);

%  \draw[line width=2pt] (80,25) [-latex] -- (140,25);
  \draw[line width=2pt] (0,25) [-latex] -- (60,25);
  \draw[line width=2pt] (160,25) [-latex] -- (220,25);
  \draw[line width=2pt] (0,-20) -- (110,-20) [-latex] -- (110,0);

% \node [left] at (10,55) {\begin{tabular}{c} Input\\Formula \end{tabular}};

  \node [left] at (10,52) {Input};
  \node [left] at (10,40) {Formula};

  \node [left] at (10, 7) {Weight};
  \node [left] at (10,-5) {Assignment};

  \node [right] at (210,52) {Weighted};
  \node [right] at (210,40) {Count};


  \node[white] at (110,37) {Weighted};
  \node[white] at (110,25) {Model};
  \node[white] at (110,13) {Counter};
\end{tikzpicture}

\end{center}

\vskip -10pt
  {\bf Given}

  \begin{itemize}
  \item CNF formula $\inputformula$ over variables $X$
    \begin{itemize}
      \bitem Set of satisfying assignments (\emph{models}) $\modelset(\inputformula)$
      \bitem Each model $\assign$ is set of literals 
    \end{itemize}
  \item Weight assignment $w(x), w(\obar{x}) \in \mathbb{Q}$ for each $x \in X$
  \end{itemize}

  {\bf Compute}
  \begin{displaymath}
  w(\inputformula) \;\; \doteq \;\; \sum_{\assign \in \modelset(\inputformula)} \;\;\prod_{\lit \in \assign} w(\lit) 
  \end{displaymath}

\end{frame}

\begin{frame}
  \frametitle{Floating-Point Cancellation}

\medskip
\textbf{IEEE Double Examples:}

\smallskip

  \begin{tabular}{rcl}
    $(10^{15} + 3.1416) - 10^{15}$ & $\longrightarrow$ & $3.125$ \\[0.5em]
%    $(10^{16} + 3.1416) - 10^{16}$ & $\longrightarrow$ & $4.000$ \\[0.5em]
    $(10^{20} + 3.1416) - 10^{20}$ & $\longrightarrow$ & $0.000$ \\[0.5em]
    $(10^{15} + 3.1416) - (10^{15} + 0.1)$ & $\longrightarrow$ & $3.000$ \\[0.5em]
  \end{tabular}

\medskip

\textbf{General Case: For $T \gg s$ and $T \approx T'$}

\medskip

  \begin{tabular}{rcl}
    $(T + s) - T$ & $\longrightarrow$ & $0$ \\[0.5em]
    $(T + s) - T'$ & $\longrightarrow$ & ?? \\
  \end{tabular}

  \medskip

  \textbf{Requirements for Cancellation}

\smallskip

  \begin{itemize}
  \item Wide dynamic range: $T \gg s$
    \begin{itemize}
      \bitem Not present in Uniform$\pm$
    \end{itemize}
  \item Homogeneity: $T \approx T'$
    \begin{itemize}
      \bitem Not present in Exponential$\pm$
    \end{itemize}

  \end{itemize}

\end{frame}

\begin{frame}
\frametitle{Interval Arithmetic}

\smallskip

\textbf{MPFI Software Library}
\begin{itemize}
\item Maintain values of form $\interval{\vmin, \vmax}$
\item $\vmin$ and $\vmax$ represented in floating-point (MPF)
\item Arithmetic operations generalized to operate on intervals
\item Guarantee approximation of true value $v^{-} \leq v \leq v^{+}$
\end{itemize}

\smallskip

\textbf{Interval Decimal Precision}

\vskip -10pt

\begin{eqnarray*}
\aerror(\interval{\vmin, \vmax}) & = & \left\{ \begin{array}{ll}
  \frac{\vmax - \vmin}{\min(|\vmin|, |\vmax|)}  & 0 \not \in \interval{\vmin, \vmax}\\[0.8em]
  0 & \vmin = \vmax = 0 \\
  1 & 0 \in \interval{\vmin, \vmax} \;\; \textrm{and} \;\; \vmin < \vmax
  \end{array} \right.\\[0.5em]
\decimalprecision(\interval{\vmin, \vmax})  & = &  \max[0, -\log_{10} \aerror(\interval{\vmin, \vmax})]
\end{eqnarray*}

\begin{itemize}
  \item Use to estimate precision of computed value $\approxw(\psi)$
\end{itemize}

\end{frame}


\begin{frame}

\frametitle{Hybrid Approach}

\smallskip

\begin{itemize}
\item decision-DNNF formula with $n$ variables
\item Want to guarantee decimal precision $D$
\end{itemize}

\medskip

\textbf{Nonnegative Weight Assignments}

\smallskip

\begin{itemize}
\item Compute required fraction size $p$
\item Use IEEE double or MPF with appropriate fraction size
  \begin{itemize}
    \bitem Test for overflow / underflow with double
    \bitem Switch to MPF when needed
  \end{itemize}
\item Theorem provides precision guarantee
\end{itemize}

\medskip

\textbf{Weight Assignments with Negative Numbers}

\smallskip

\begin{itemize}
\item Start with fraction size $p$ based on nonnegative case
\item Use MPFI with increasing fraction sizes
\item Use MPQ as last resort
\end{itemize}

\end{frame}

\begin{frame}

\medskip

\frametitle{Hybrid Benchmark Cumulative Evaluation Time}

\begin{tikzpicture}[scale = 0.80]
  \begin{axis}[
      ybar stacked,
      width = 12cm,
      height=8cm,
      bar width=15pt,
%%      nodes near coords,
%%      enlargelimits=0.15,
      legend style={at={(0.42,-0.30)},
        anchor=north, legend columns=-1},
      ylabel = {Total Time (hours)},
      xlabel = {Target precision $D$},
      ymax = 30,
      ytick = {5, 10, 15, 20, 25, 30},
      yticklabels = {5, 10, 15, 20, 25, 30},
      symbolic x coords={1, 5, 10, 15, 20, 25, 30, 35, 40, 45, 50, 55, 60, 65, 70},
      xtick=data,
      ]
\input{data-formatted/tabulate-effort}
\legend{\strut ERD, \strut MPF-64, \strut MPF-128, \strut MPF-256, \strut MPFI-64, \strut MPFI-128, \strut MPFI-256, \strut MPQ}
  \end{axis}
\end{tikzpicture}
  
\end{frame}


\end{document}
